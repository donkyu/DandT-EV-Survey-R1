
%%%%%%%%%%%%%%%%%%%%%%%%%%%%%%%%
\section{Propulsion Power Modeling, Estimation, Runtime Optimization, and Design-Time Optimization} \label{sec:propulsion}
%%%%%%%%%%%%%%%%%%%%%%%%%%%%%%%%

Electric vehicles of course consume majority of electrical energy for propulsion. Electric vehicle propulsion energy generally increases by their weight but also largely variable by the driving conditions such as road slope. Unfortunately, electric vehicles are not lighter than the similar scale of internal combustion engine vehicles due to the battery weight while production electric vehicles deploy light materials such as aluminium to a large portion of chassis and bodies to compensate the battery weight. For instance, Chevrolet Bolt (164” long) weighs 3563 lbs, but a Honda Civic (177” long) does ony 2752 lbs.

Electric powertrain efficiency is already close to the their theoretical limits. Electric vehicles also have advantages in well-to-wheel efficiency from idling, downhill driving and braking using regenerative braking. Nevertheless, the well-to-wheel efficiency is not much higher than that of internal combustion engine vehicle. Regenerative braking is fundamental advantage of electric vehicles that reclaim kinetic energy back to electrical energy, which is dissipated to heat energy in the internal combustion engine vehicles. However, modern internal combustion engine vehicles adopt engine stop-and-go during idle and regenerative braking using the alternator.

One need to obtain the actual model coefficients for actual deployment of the power models of~\eqref{eq:traction_model}. Vehicle manufacturers often provide a part of model coefficients in the specification sheet such as the aerodynamic resistance as well as the curb weight, and tire manufactures commonly advertises the rolling resistance values of their products. Wind chamber experiments can also obtain unknown aerodynamic resistance values.

%%%%%%%%%%%%%%%%%%%%%%%%%%%%%%%%
\subsection{Propulsion Power Modeling} \label{subsec:propulsion_model}

% Come from section {sec:driving_profile}
Electric Vehicle (EV) as a multi-domain Cyber-Physical System (CPS) consists of two major systems of electric motor and auxiliary system that are the main contributors to the power consumption~\cite{AF_1,AF_2,AF_3,Park:DAC13}.

\textbf{Electric motor} is the main system that generates enough force to propel the vehicle at a desired speed and acceleration. The electric motor consumes the energy stored in a storage such as a lithium-ion battery for generating the driving power and force. The electric motor can also behave as a generator; hence, the electric motor can generate electric power while generating driving power and force in the opposite direction. This regenerative braking system helps the EVs achieve higher energy efficiency and extended driving range~\cite{AF_5,AF_6,AF_7}.

\textbf{Propulsion power} generated or consumed by the electric motor significantly influences the stored battery energy, the EV driving range, and the battery lifetime~\cite{AF_8,AF_9,AF_10}. Hence, modeling, estimation, and optimization of the propulsion power is a major solution to extending the driving range and battery lifetime.


\subsubsection{Physical Model} \label{subsubsec:physical_model}

Most of all, it is very important to have accurate propulsion power modeling for power estimation, runtime and design-time power optimization. Vehicle power consumption modeling has been used for eco-driving research for engine vehicle research over decades. An underlying propulsion power model is based on well-known physics equations. Such vehicle dynamics model basically takes into account the propulsion force applied to the tires.

\begin{equation} \label{eq:traction_model}
P_{trac} = F \frac{ds}{dt} = Fv= (F_{R} + F_{A} + F_{G} + F_{I} + F_{B}) v
\end{equation}
%
where $F_R$, $F_A$, $F_G$, $F_I$, and $F_B$ are the rolling resistance, aerodynamic resistance, gradient resistance, inertia resistance, and brake force provided by hydraulic brakes, respectively~\cite{Chang:ICCAD14, Park:DAC13}. This model applies to any type of propulsion systems as it assumes 100\% efficiency of the propulsion system. As a result, this physics model does not explain electric-vehicle-specific power consumption behaviors such as electric traction motor characteristics, regenerative braking, battery efficiency, etc. Such simplification still can marginally estimate electric vehicle propulsion power but hardly use for propulsion power saving research.

\subsubsection{Motor Efficiency} \label{subsubsec:motor_eff}

The primary discrepency between the electric vehicles and internal combustion engine vehicles can be said the source of the propulsion force: electric motors and internal combustion engines. Adding the electric motor characteristics to the physics model in~\eqref{eq:traction_model} is able to accommodate electric powertrain characteristics.

\begin{figure}
\centering
\includegraphics[width=1.0\hsize]{Figures/Naehyuck_Chang/motor_engine_torque_map.pdf}
\caption{A 1.4 liter GM Voltec Internal combustion engine efficiency map (a) and a Nissan Leaf 80kW electric motor efficiency map (b)~\cite{Nissan:SAE11, Grebe:ATZ11}.}
\label{fig:torque_map}
\end{figure}      

Figure~\ref{fig:torque_map}(a) illustrates efficiency maps of a 1.4 liter GM Voltec engine and a Nissan Leaf 80kW motor. Electric vehicle efficiency is largely different from that of engine vehicles under the same driving condition as the source of the traction force is a completely different characteristics, even if electric vehicles are subject to~\eqref{eq:traction_model} as well. Electric motor efficiency varies by the motor type, motor revolution per minute (RPM) and torque. Consideration of traction device efficiency can significantly enhance the electric-vehicle-specific characteristics over~\eqref{eq:traction_model} as shown as

\begin{align} \label{eq:motor_eff_model}
P_{EV} &= \frac{P_{trac}}{\eta_{EV}} \\
\eta_{EV} &= \frac{P_{trac}}{{P_{trac} + k_c T^2 + k_i \omega + k_w \omega^3 + C}}, \nonumber
\end{align}
%
where $k_c$, $k_i$, $k_w$, $C$ are a copper loss, a iron and friction loss, a windage loss and constant loss, respectively~\cite{Vaz:JPS14,Lin:CCA14,Oliva:PHM13,Kachroudi:TVT12}.

\subsubsection{Drivetrain Efficiency} \label{subsubsec:drive_eff}

Power model of~\eqref{eq:motor_eff_model} reflects efficiency of the electric traction motors. The actual electric vehicle propulsion efficiency is not only dependent on the traction motors but affected by the drivetrain loss from the gearbox, axle bearings, driveshaft, gearbox fluid, and many more, which varies by the vehicle speed, acceleration and motor torque. The actual model coefficients can be obtained from the motor companies, but this is not always the case.
It is hard to rely on vehicle specification sheets to figure out drivetrain efficiency. Testrun in various conditions helps collect range of power consumption data and obtain the model coefficients by a curve fitting method~\cite{Dib:OGST12} with a high model fidelity of 1.7\% overall cumulative energy consumption error\cite{Dib:OGST12}. Dynamometer test provides an isolated experimental environment~\cite{Schwickart:JFI15} to obtain the actual efficiency of the traction motor combined with the drivetrain.

Consideration of accurate efficiency change of the traction motors and drivetrain loss can be highly nonlinear, and thus it can be challenging to use mathematical equation models such as~\eqref{eq:traction_model} and~\eqref{eq:motor_eff_model}. Lookup tables are convenient to describe complicated power characteristics and also easy to implement a power simulator~\cite{Schwickart:JFI15} while less convenient to use for systematic optimization.

More precise electric vehicle power modeling requires extensive power measurement and characterization. Use of production electric vehicles has various limitations in non-intrusive power measurement. Fabrication of electric vehicles~\cite{Hong:ASPDAC16} or electric vehicle conversion of a production internal combustion engine vehicle~\cite{Wu:TR15} efficiently overcomes the limitation in non-intrusive power measurement. The battery power should be measured together with the vehicle driving condition. Global positioning system (GPS) provides the location, altitude and thus rough road slopes. Such whitebox experiments can efficiently isolate the propulsion power consumption from non-propulsion power consumption and give a lot of benefits in power modeling~\cite{Hong:ASPDAC16,Wu:TR15}. It turns out that drivetrain loss in the electric vehicle is proportional to the square of the vehicle speed while the electric powertrain model only has a linear term of the speed as shown in~\eqref{eq:drive_eff_model}~\cite{Hong:ASPDAC16}.

\begin{align} \label{eq:drive_eff_model}
P_{EV-specific} &= \frac{P_{trac}}{\eta_{EV-specific}} \\
\eta_{EV-specific} &= \frac{P_{trac}}{{P_{trac} + C_0 + C_1 v + C_2 v^2 + C_3 T^2}} \nonumber
\end{align}
%
where $C_0$, $C_1$, $C_2$, and $C_3$ mean coefficients for constant loss, iron and friction
losses, drivetrain loss, and copper loss, respectively. These methods guarantee a very high estimation accuracy (e.g., 2.89\% maximum absolute error in~\cite{Hong:ASPDAC16}) for the target vehicle model. However, these models can only provide limited accuracy when applied to general electric vehicles.

\subsubsection{Electric Vehicle Driving Range Estimation} \label{subsubsec:range_est}

Most electric vehicle owners have range anxiety due to the limited driving range and inaccurate remaining range gauge on the instrument cluster. Depletion of the vehicle battery in the middle of a trip may bring a trouble equivalent to vehicle breakdown~\cite{Hong:ASPDAC16}.

The electric vehicle power models of~\eqref{eq:traction_model}-\eqref{eq:drive_eff_model} require vehicle weight, acceleration, speed, road slope as the input variables. It is impossible to know the actual values of the future vehicle acceleration and speed in real cases due to the uncertainty in the driving conditions. Such unknown factors can be modeled as probability density functions, and the resultant driving range estimation can also be represented probabilistically. This work models the uncertainty with a statistical speed profile~\cite{Oliva:PHM13}.

Consideration of the battery characteristics is also attempted such as the state of charge versus the battery output voltage. For example, battery current keeps increasing as battery discharges even if the vehicle maintain a constant cruising speed on a flat road~\cite{Vaz:JPS14}. This work assumes that the vehicle velocity should be decreased as the battery voltage decreases, which is not in the real case because the battery should be cut off before it becomes fully discharged (over-discharge protection, typically at 20\% SoC) by the battery management system (BMS), and the motor driver should be able to maintain the same torque within the legal voltage range of the battey.

The fidelity of the power consumption model directly impacts on the range estimation accuracy~\cite{Hong:ASPDAC16}. The power model in~\eqref{eq:drive_eff_model} that considers the drivetrain efficiency exhibits less than 2.89\% maximum absolute error in the power consumption, and thus between 0.05\% to 4.39\% in range estimation compared with real measurement~\cite{Hong:ASPDAC16}. On the other hand, the power models in~\eqref{eq:traction_model} and~\eqref{eq:motor_eff_model}, which consider only propulsion force and motor efficiency, obtain 6.85\% and 9.33\% maximum absolute error in the power consumption and up to 19.19\% and 21.75\% in range estimation, respectively.

%%%%%%%%%%%%%%%%%%%%%%%%%%%%%%%%
\subsection{Runtime Propulsion Power Optimization} \label{subsec:propulsion_runtime_opt}

A large portion of previous work relies on the physics model in~\eqref{eq:traction_model} or adding the motor efficiency in~\eqref{eq:motor_eff_model} to estimate electric vehicle power consumption. Therefore, derivation of the energy-optimal vehicle speed for a given road slope is often attempted with~\eqref{eq:traction_model} or~\eqref{eq:motor_eff_model}~\cite{Wu:TR15,Vaz:JPS14,Lin:CCA14,Oliva:PHM13,Kachroudi:TVT12}.
As mentioned in Section I, electric vehicles consume different amount of power and energy by the vehicle speed and acceleration due to the non-constant efficiency of the traction motor and the drivetrain even if the vehicles drive on the same route with the same payload. This inspires runtime propulsion power optimization, which is a similar concept of eco-driving of internal combustion engine vehicles~\cite{Kamal:TITS11,Ozatay:IFAC14,Ozatay:TITS14,Khayyam:ESA12}. However, the fundamental discrepancy in the electric powertrain from the engine powertrain makes the optimization policy largely different.

\subsubsection{Propulsion Power Optimization with a Constant Road Slope} \label{subsubsec:opt_const_slope}

According to the electric vehicle power model in~\eqref{eq:drive_eff_model}, propulsion power varies by the road slope as well as the vehicle speed, and the total energy consumption to finish the same distance of travel, i.e., the integration of the vehicle power over the driving time, forms a convex function of the vehicle speed for a given constant road slope as shown in Fig.~\ref{fig:energy_vs_vel}. The shape of the convex function varies by the road slope, which implies that there exists the energy-optimal electric vehicle speed by a given road slope.

\begin{figure}
\centering
\includegraphics[width=1.0\hsize]{Figures/Naehyuck_Chang/energy_consumption_by_velocity.pdf}
\caption{Energy consumption versus electric vehicle velocity on various road slopes.}
\label{fig:energy_vs_vel}
\end{figure}      

It is very interesting to note that the speed increases by the road slope in the case of electric vehicles, i.e., electric vehicles should run faster on a steeper road to save energy. This inspires the minimum-energy velocity planning over time (or distance) for a given driving mission: a route, a payload and a deadline, which is electric-vehicle-specific. Therefore, one of the simplest propulsion power optimization problem is defined as finding the energy-optimal vehicle speed planning (acceleration, cruding and deceleration) for a given constant road slope.
Related previous work derives the energy-optimal vehicle speed planning without considering the traffic condition or fast computation assuming a constant road slope~\cite{Yan:NAPS14,Dib:CEP14}. A trapezoidal driving, a combination of a constant acceleration, a constant speed cruising and a constant deceleration, is not an impractical driving method in a real world with a smooth traffic flow, and a fast computation for a series of trapezoidal driving patten planning can accommodate intersections with stop signs~\cite{Yan:NAPS14}. Fast computation is always beneficial for practical use, which can eventually accommodate traffic condition and route change guided by GPS navigation systems. A simplified analytical method that ignores aerodynamic resistance and motor efficiency can rapidly solve the energy-aware trapezoidal driving problem~\cite{Dib:CEP14}.

\subsubsection{Propulsion Power Optimization with a Variable Road Slope} \label{subsubsec:opt_variable_slope}

Many cities are located on hill as well as on flat lands. The energy-optimal trapezoidal driving may not work if the road slope varies before a single trapezoidal driving ends. The trapezoidal driving with a constant cruising speed is no longer energy optimal if the road slope changes. Consideration of the road slope change is crucial for freeway driving where there is no traffic signal nor intersection with stop signs.

The problem to solve the propulsion power optimization with a variable road slope is formulated as an optimal velocity selection problem over distance as shown in Fig.~\ref{fig:formulation}(a). The X-axis and Y-axes in Fig.~\ref{fig:formulation}(b) denote the driving distance from the starting point and the driving velocity, respectively. Each node indicates the driving velocity at a given distance, and an edge between two nodes implies acceleration or deceleration during the distance step.

\begin{figure}
\centering
\includegraphics[width=1.0\hsize]{Figures/Naehyuck_Chang/Discretization.pdf}
\caption{Problem formulation for the propulsion power optimization with a variable road slope.}
\label{fig:formulation}
\end{figure}      

The energy-optimal speed profiles exhibit dynamically changing cruising speed taking the regenerative braking and speed limit as well. Previous work tackles this problem without considering the traffic condition and stop signs by the use of analytical method~\cite{Kamal:TITS11,Ozatay:IFAC14}, dynamic programming~\cite{Ozatay:TITS14}, machine learning~\cite{Khayyam:ESA12}, and model predictive control~\cite{Schwickart:JFI15}. Model predictive control (MPC) is also used to derive the optimal motor torque with the objective of a weighted sum of the energy consumption in every 400 m drive and the kinetic energy difference between the best constant velocity and the derived velocity~\cite{Schwickart:JFI15}. These methods derives the energy-optimal speed profiles on a variable-slope road with a reasonable computational complexity but can hardly guarantee the deadline constraint.

Even if dynamic programming can hardly guarantee a hard deadline while minimizing the driving energy, making the cost function as a weighted average of the driving time and driving energy can consider the deadline in a certain degree~\cite{Dib:IVPPC11,Dib:OGST12,Mensing:TR13,Lin:CCA14}. This method is useful for soft real-time system, which is the case of most common driving missions.

The dynamic programming with an weighting factor is formulated with follow:

\begin{align} \label{eq:dynamic_programming}
min ~& \sum_{d=1}^{M}\sum_{v=1}^{N}\{w \Delta t x(d,v)P_{d,v} + (1-w)\Delta t \}\\
Subject~to ~& \sum_{v=1}^{N}x(d,v) = 1~~~\forall v \nonumber \\
~&x(d,v) = \{0, 1\} ~~~\forall d, v \nonumber
\end{align}
%
where $\Delta t$ is a driving time during the distance step, $d$ is a distance from the start point, $v$ is a velocity at the distance, $P_{d,v}$ is a power consumption at a distance $d$ and a velocity $v$, and w is weighting factor. $\Delta t$ is varied by the driving velocity and the acceleration during the distance step.

A more explicit deadline awareness can be derived by iterative methods or semi-exhaustive heuristics that forms a multi-objective optimization problem such as genetic algorithm~\cite{Kachroudi:TVT12,Dovgana:ASC14}. However, these methods are subject to a very high computational complexity, which can be hardly used for online methods that are reactive to the traffic condition change and unexpected interruption by other cars or pedestrians.

\subsubsection{Propulsion Power Optimization Considering Traffic and Intersections} \label{subsubsec:opt_traffic}

The energy-optimal driving method introduced in the previous sections may not actually feasible in real traffic conditions. Safety is always first than energy saving, and thus heavy traffic often makes the vehicle slow down than the energy-optimal speed. Such traffic condition cannot be fully known a priori, and the derived energy-optimal driving method should be computed again whenever unexpected disturbance happens, which ends up with a local minimum energy consumption. Consideration of the traffic condition and traffic signal makes the energy-optimal driving more realistic~\cite{Mensing:TR13,Lim:TVT17,Mandava:ITSC09,Asadi:TCST11,Ozatay:CDC13,Nunzio:JRNC15,Wu:ITS15}  even if such work is still based on strong assumptions. Such work is rather a high-level driving problem rather than electric vehicle-specific driving energy optimization.

Average speed of preceding vehicles on the same route marginally explains the distance between vehicles, which constrains the maximum possible target vehicle speed~\cite{Mensing:TR13,Lim:TVT17}. Traffic signals are often controlled by a periodic schedule that mimics the energy-optimal driving~\cite{Mandava:ITSC09,Asadi:TCST11,Ozatay:CDC13,Nunzio:JRNC15,Wu:ITS15}.

A majority portion of intersection driving management is to optimize the throughput. There are intersection driving management work that explicitly optimize driving energy~\cite{Nunzio:JRNC15,Wu:ITS15} demonstrating 8\% to 48\% energy saving. Once again, such work does not perform electric-vehicle-specific driving energy optimization, and consideration of electric powertrain and regenerative braking may exhibit a large difference in the energy consumption.

%%%%%%%%%%%%%%%%%%%%%%%%%%%%%%%%
\subsection{Design-Time Propulsion Power Optimization} \label{subsec:propulsion_design_time_opt}

It would be a more proactive energy saving if the energy optimality is considered at design time. Determination of the battery capacity is a really critical factor in electric vehicle unlike internal combustion engine vehicles questioning the gas tank size. The battery capacity directly impacts on the vehicle weight as well as the driving range, vehicle price and space utilization. As observed in Sections (modeling and runtime optimization), vehicle weight is a major variables that determines energy efficiency. Installation of a larger-capacity battery certainly increase the driving range but certainly decreases fuel efficiency (MPGe, mile per gallon gasoline equivalent.) Electric vehicles would better not carry an unnecessarily big battery pack while commuting in a short distance. A heavier battery pack also degrade the performance of the vehicle. On the other hand, a large capacity battery pack provides clear benefits such as a larger C value that makes the battery efficiency higher, a lower battery temperature and a longer cyclelife. As a result, determination of the battery pack capacity must consider various aspects. To make a long story short, tailored specification of an electric vehicle dedicated to a particular operating scenario gives a huge opportunity to save energy consumption.

\begin{figure}
\centering
\includegraphics[width=1.0\hsize]{Figures/Naehyuck_Chang/design_optimization.pdf}
\caption{Design-time propulsion power optimization dedicated to a driving mission.}
\label{fig:design_opt}
\end{figure} 

Design-time optimization requires extensive evaluation of the vehicle characteristics whenever a design variable is changed. It is not feasible to perform actual driving test during the design time. Automotive original equipment manufacturer (OEM) collected range of data over decades and attempt to find a suboptimal setup. Instead, simulation environment can significantly reduces the design effort. Driving energy simulators evaluate the power consumption by changing the powertrain specification and estimate the vehicle performance and driving range as shown in Fig.~\ref{fig:design_opt}. A vehicle simulator on a Matlab can evaluate the energy consumption by the motor, battery characteristics, gearbox, vehicle chassis, and so on~\cite{Butler:TVT99,Markel:JPS02}. Related previous work optimizes the battery pack capacity for a given driving mission~\cite{Gao:TIE10}. A supercapacitor and battery hybrid can prolong the battery lifetime~\cite{Song:AE14}. The optimal reduction ratio of the traction motor to the wheels also increases the electric vehicle fuel efficiency~\cite{Yin:ITEC-AP14}. Power rating of the traction motor, gear reduction ratio and the battery capacity should be jointly optimized. However, such design space is very wide, and evaluation time complexity is huge. Genetic algorithm is a common way to solve such multi-objective optimization, and related work analyzes vehicle efficiency by the motor power rating and the battery capacity~\cite{Pozo:AMAA15}. The design space becomes limited if the available components are in the limited number of libraries. Related work derives the optimal hybrid electric bus with 19 choices of diesel engine power rating, 19 choices of alternator capacites, 28 different size of traction motors, nine battery pack capacity variations, seven gearbox reduction ratios~\cite{Hasanzadeh:INDEL14}. Similar work attempts the best configuration of passenger vehicles and pickup trucks by the driving scenarios with four fuel cells, four motor power ratings, and eight battery pack capacities using Genetic algorithm~\cite{Ribau:AE14}.

