
%%%%%%%%%%%%%%%%%%%%%%%%%%%%%%%%
\section{Introduction}
%%%%%%%%%%%%%%%%%%%%%%%%%%%%%%%%

Electric vehicles have many fundamental advantages over internal combustion engine vehicles. Electric vehicle idea is not new. As early as in 1891, William Morrison of Des Moines, Iowa already built the first successful electric vehicle in the United States. However, internal combustion engines almost completely replaced electric powertrains in the production vehicles for the past 100 years. 

Thanks to recent evolution of Lithium-ion batteries and strong demand for zero-emission vehicles, electric vehicles have successfully repositioned demonstrating commercial success. Nevertheless, electric vehicles still have challenges in terms of the battery weight, fully-charged driving range, costs, etc. In this survey, we provide range of information toward “low-power” electric vehicles that is able to inspire how design automation can overcome such primary electric vehicles’ disadvantages more systematically and more efficiently. 

We first introduce propulsion power modeling, estimation, runtime optimization, and design-time optimization in Section~\ref{sec:propulsion} as majority of battery energy in the electric vehicles is used for propulsion. 
%
%Section~\ref{sec:driving_profile} introduces driving profile modeling, estimation and runtime optimization on the top of Section~\ref{sec:propulsion}. Driving profile also largely impacts on non-propulsion power consumption. Indeed, modern electric vehicles also consume significant amount of battery energy for non-propulsion power such as heating, ventilation, and air conditioning (HVAC.) 
%
Section~\ref{sec:driving_profile} introduces driving profile modeling, estimation and runtime optimization on the top of Section~\ref{sec:propulsion}. 
%
Driving profile also largely impacts on non-propulsion power consumption. Indeed, modern electric vehicles also consume significant amount of battery energy for non-propulsion power such as heating, ventilation, and air conditioning. Section~\ref{sec:non-propulsion} summarizes non-propulsion power modeling, estimation and runtime optimization.
%
As battery systems are the key component for electric vehicles, we introduce modeling, simulation, estimation, and optimization of electric vehicle battery systems in Section~\ref{sec:battery}. 
%
Energy harvesting is a proactive way to manage electric vehicle energy. Section~\ref{sec:harvesting} also summarizes vehicle energy harvesting methods including solar energy, wind energy, regenerative shock absorbers, and regenerative brakes. 
%
Section~\ref{sec:power_grid} summarizes how electric vehicle charging impacts on the power grids. This section also points out unsolved issues and future research direction.


