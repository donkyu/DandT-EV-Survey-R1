%%%%%%%%%%%%%%%%%%%%%%%%%%%%%%%%%
%\section{Battery Systems for Electric Vehicles} \label{sec:battery}
%%%%%%%%%%%%%%%%%%%%%%%%%%%%%%%%%
%
%%\abstract
%Battery systems are an important part of electric vehicles.  The main research areas in battery system are modeling, simulation, estimation and optimization. In the following, literature survey of battery systems for electric vehicles is presented. 
%
%%%%%%%%%%%%%%%%%%%%%%%%%%%%%%%%%
%\subsection{Modeling}
%
%Rechargeable batteries that offer high energy and
%power density and relative safety are the choice for many applications from consumer
%electronics to electric vehicles. One example of such batteries are Lithium-ion batteries. Modeling the cycle and
%calendar life of such batteries is needed to set
%expectations of reliable product performance. Generally, there are two types of models to predict the battery life~\cite{ZS_millner}. One type of models describe chemistry of the cell~\cite{ZS_Ning}. These models, with the help of extensive computation facilities, provide time and temperature effects accurately but do not match with cycle life test data. Such models are best suited for
%optimization of the physical design aspects of electrodes
%and electrolyte~\cite{ZS_den}. The other type is empirical models~\cite{ZS_dubarry} that fit the test data to equivalent circuit models using power law relationships. These models have limited range of parameters and require extensive testing of each type of cells. 
%
%A simple battery model consists of an ideal battery with an open circuit voltage and and a constant internal series resistance. A new battery model based on simple battery model was proposed in~\cite{ZS_jean}. In this battery model, SoC (State of Charge) of the battery is considered as well by making the internal resistance variable. Another commonly used model is the Thevenin battery
%model, which consists of an ideal no-load battery voltage ($E_o$),
%internal resistance ($R$), capacitance ($C_o$) and overvoltage
%resistance ($R_o$). $C_o$ represents the capacitance of the parallel
%plates and $R_o$ represents the non-linear resistance contributed
%by the contact resistance of plate to electrolyte. The main disadvantage of the Thevenin battery model is that
%all the elements are assumed to be constant, but in fact all the
%values are functions of battery conditions.
%
%
%\begin{figure}[b]
%\centering
%\includegraphics[scale=0.5]{Figures/Zili_Shao/ZS_figure1.pdf}
%\caption{Thevenin battery model.}
%
%\end{figure}
%
%An empirical methamatical model is developed in \cite{ZS_jayne,ZS_sims}. The improvement of this model is to account for the nonlinear
%characteristic of both the open circuit voltage and
%internal resistance. A sophisticated and accurate dynamic model for simulation purpose was proposed in \cite{ZS_gig}. This model includes electrolyte reaction, Ohmic effect and leakage capacitance as well as the self discharge current. The drawbacks of this model include longer time for computation because of its higher order and complicated modelling procedure that involves a lot of empirical data. Another battery model is called over-current battery model as described in \cite{ZS_rob}. It has a variable
%current source, a variable voltage source, a variable resistor, and a capacitor. 
%This model provides a good representation of both variable
%internal drop in the battery and changes in the output voltage
%due to the SoC. However, one of its drawbacks is
%that too many parameters are required. A model described in \cite{ZS_zi,ZS_mar}
%considers most nonlinear battery element characteristics both during charging and 
%discharging as well as their dependence on the SoC of the battery. All the elements included in this model are functions of the
%open-circuit voltage of a battery, which in turn relates to state of
%charge~\cite{ZS_chan}.
%
%
%The model in \cite{ZS_chen} is capable of predicting run time and I-V performance, but is not
%accurate for the transient response to short-duration loads. As a result, it does not predict accurately
%the SoC throughout drive cycles for electric-vehicle-related simulations \cite{ZS_kro}. A model proposed in \cite{ZS_kro} describes accurate determination of the discharge capacity, which is a function of the discharge rate, temperature, cycle number, and a rate factor that accounts
%for a decrease in capacity due to unwanted side reactions
% as the current increases. This model can also predicts SoC, terminal voltage and power losses of various batteries for electric vehicles. Along with predicting the battery behaviour accurately, this model can be used for simulation as well. 
% 
%%%%%%%%%%%%%%%%%%%%%%%%%%%%%%%%%
%\subsection{Simulation}
%
%Simulation with an accurate battery model is able to analyze complex phenomena. Average ratings can be
%derived from steady-state operation and characteristics of the
%system, but peak ratings can only be estimated from the steady state results and are often inadequate. A detailed model is appropriate for short-term
%analysis and supplements the long-term capabilities of a
%steady-state model as described in \cite{ZS_wip,ZS_adv}.  In order to better understand losses, thermal
%characteristics, and durability of various  electric vehicle batteries, a
%dynamic simulator is a key tool. Estimates of conducting losses in
%power electronics devices, batteries, supercapacitors, and other components can be supported directly with a dynamic model. 
%%The ADVISOR program developed through the National Renewable Energy Laboratory (NREL) \cite{ZS_bai} is a simulator based on static maps that reflects steady-state behavior of vehicle subsystems. 
%Dynamic models are needed to make lower-level comparisons
%among subsystems and support subsystem design. Steady-state simulation tools for the design and analysis of
%hybrid electric automotives have been developed in recent years and support comparisons over long drive cycles. In the past, dynamic
%simulation models have focused mainly on the analysis of
%control strategies \cite{ZS_amr}. A dynamic simulator in \cite{ZS_log} offers a more
%detailed look, based on dynamic equations of each
%subcomponent (the engine, battery, inverter, motor, and
%transmission) on microsecond time scales.
%
%Several simulation systems have been
%developed to describe the operation of hybrid electric power
%trains such simple electric vehicle simulation (SIMPLEV) from
%the DOE’s Idaho National Laboratory \cite{ZS_col}. Simulation programs include MARVEL from
%Argonne National Laboratory \cite{ZS_wal}, CarSim from AeroVironment
%Inc., JANUS from Durham University \cite{ZS_bum}, 
%%ADVISOR from the DOE’s National Renewable Energy Laboratory \cite{ZS_wip},
%Vehicle Mission Simulator \cite{ZS_noo}, and others \cite{ZS_aue,ZS_kri}. A
%simulation model (ELPH) is used to study the viability of an electrically
%peaking control scheme and to determine the applicability of
%computer modeling to electric vehicle design \cite{ZS_bun}. V-Elph \cite{ZS_ste,ZS_but} is a system-level modeling, simulation,
%and analysis package. This package uses Matlab/Simulink to study issues related to electric vehicle and
%hybrid electric vhicle design such as energy efficiency, fuel economy, and
%vehicle emissions. V-Elph facilitates in-depth study of power
%plant configurations, component sizing, energy management
%strategies, and the optimization of important component parameters
%for several types of hybrid or electric configuration
%or energy management strategies with visual programming
%techniques, allowing the user to quickly change architectures,
%parameters, and to view output data graphically \cite{ZS_meh}.
%
%%%%%%%%%%%%%%%%%%%%%%%%%%%%%%%%%
%\subsection{Estimation}
%
%Automotive Lithium-ion batteries have high capacity and large serial parallel
%numbers, which, coupled with the problems such as safety,
%durability, uniformity and cost, imposes limitations on the wide
%application of lithium-ion batteries in vehicles. Lithium-ion
%batteries must operate within the safe and reliable operating area, which is restricted by temperature and voltage windows. Exceeding the restrictions of these windows will lead to rapid
%attenuation of battery performance and even result in safety
%problems \cite{ZS_lu}. The accommodation
%of such operating conditions requires that a management system has accurate knowledge of  many factors e.g. the  SoC to facilitate safe and efficient operation.
%Failure to control SoC, which may lead to under- or over-charging
%conditions, can degrade the ability of the pack to source/sink
%subsequent power transients \cite{ZS_bha}.
%
%A variety of techniques have been proposed to measure or
%monitor the SoC of a cell or battery, each having its relative
%merits \cite{ZS_pil}. Coulomb counting or current
%integration is, at present, the most commonly used technique,
%requiring dynamic measurement of the cell/battery current,
%the time integral of which is considered to provide a direct
%indication of SoC \cite{ZS_cau}. 
%
% Due to complexity of electrochemical processes in batteries and noise in sensors, many sophisticated algorithms have been proposed for efficient battery monitoring, such as estimation of SoC \cite{ZS_cun}. An  earlier  work  which  uses  voltage  as  the  basis  of  SoC estimation is presented in \cite{ZS_ver}. Though this work deals with complexities  such  as  hysteresis,  it  is difficult to estimate SoC for some battery types such as Nickel-metal  hydrid.  A  fuzzy  logic  approach  for  SoC  estimation is  presented  in \cite{ZS_sin}. This approach  requires  training  data, which is  challenging  because  of  changing  properties  in  different conditions.  A  complex  neural-network-based  approach  is presented in \cite{ZS_ang}, but it is restricted to lead acid batteries and requires complex network design and computation. 
% 
% A hybrid neural-network and genetic-algorithm based approach of SoC estimation  of  series  connected  modules  (battery  cells)  is discussed in \cite{ZS_11}. Despite the promising result, the proposed method is relatively complex and has high computational cost. In \cite{ZS_12}, a complicated mathematical model has been devised, specifically for lead acid batteries, which can predict the SoC and remaining operation time with up to 10\% error. A nonlinear estimation method based on the sliding mode observer is presented in \cite{ZS_13}. This work also discusses the parameters of the  battery  model  through  different  tests.  A  novel  algorithm is  presented  in \cite{ZS_14}, which  combines  the  weighted  sum  of complex  voltage  based  methods  and Coulomb  counting  techniques.  An  earlier  method  based  on Kalman  filter  is  presented  in \cite{ZS_15,ZS_16,ZS_17,ZS_18}.  In  addition  to  a few similar methods, a famous estimation algorithm based on Extended Kalman Filter is proposed in \cite{ZS_19,ZS_20,ZS_21,ZS_22,ZS_23,ZS_24} and has shown promising results. The method proposed in \cite{ZS_shaheer} can robustly estimate SoC by using simple but accurate battery models employing a conservative filtering technique. The $H_\infty$ filter is  solved  optimally  by  formulating  it  as  a Linear Matrix Inequality problem. The separation of computation (calculation of gain once and iterative implementation of estimator) enables the proposed method to be implemented on embedded controllers without compromising real-time operation requirements.
%
%SoH (state of Health) is a quantifying estimation that reflects the general
%condition of a battery and its ability to deliver the specified
%performance compared with a fresh battery. Battery
%capacity (i.e., the energy storage capacity) has long been the
%target of researchers as the definitive battery SoH indicator \cite{ZS_yu}. In general, a lithium battery is deemed to fail when its capacity
%fades by 20 \% of the rated value \cite{ZS_pas}.
%
%%%%%%%%%%%%%%%%%%%%%%%%%%%%%%%%%
%\subsection{Optimization}
%
%One of the problems for electric vehicle designers is that energy and power density need to be carefully balanced. Heavy batteries increase the vehicle’s range and light batteries increase the power-to-weight ratio, enabling the electric vehicle to achieve better acceleration.
%More batteries, however, increase the cost of the vehicle design. 
%
%Electric vehicles do have quick acceleration, due to the high torque of the electric motor. Currently nickel metal hydride (NiMH) and Lithium-ion are the common, though research continues into new battery chemistries, such as nano-phosphate cathodes, to allow for high discharge rates and higher performance. The discharge rate for batteries contributes to not only acceleration rates, but also how quickly the batteries will be charged. The higher the rate, the more energy is lost through heat. Supercapacitors are available as alternative components. They have high charge and discharge rates, resulting in lower losses but these benefits can be outweighed by their low energy density \cite{ZS_coi}.
%
%The charging management concepts can also be divided into
%centralized and decentralized approaches \cite{ZS_war}. The decentralized
%approaches let the electric vehicles optimize its charging behavior
%based on, for example, a price signal broadcast. The drawback
%of this approach is that the electric vehicle needs to collect and store the
%trip history. If the electric vehicles should coordinate their charging, for
%example to include distribution grid constraints, the need for
%V2V (Vehicle to vehicle) communication is high. 
%
%The centralized approaches
%focus on a centralized unit that directly controls the charging of electric vehicles. Additional study on forecasting and managing electric vehicle
%charging can be found in \cite{ZS_rah,ZS_gal}. A novel method for optimization is presented by \cite{ZS_sun} which describes the method for planing the charging of electric vehicles with electric grid constraints including voltage and power. This method provides an individual charging plan for each vehicle, which helps in avoiding the congestion on distribution grids. Using quadratic approximation approachs to optimize the behaviour  electrical vehicle battery, the goals of minimizing charging costs, achieving satisfactory state of energy levels and optimal power balancing have been achieved \cite{ZS_sun2}.

