

\bibitem{ZS_millner}
Alan Millner, "Modeling lithium ion battery degradation in electric vehicles." Innovative Technologies for an Efficient and Reliable Electricity Supply (CITRES), 2010 IEEE Conference on. IEEE, 2010.

\bibitem{ZS_Ning}
Gang Ning, Ralph E. White and Branko N. Popov. "A generalized cycle life model of rechargeable Li-ion batteries." Electrochimica acta 51.10 (2006): 2012-2022.

\bibitem{ZS_dubarry}
Matthieu Dubarry et al. "Capacity loss in rechargeable lithium cells during cycle life testing: The importance of determining state-of-charge." Journal of Power Sources 174.2 (2007): 1121-1125.

\bibitem{ZS_jean}
Jean Paul Cun, Jean No Fiorina, Michael Fraisse, And Henri Mabboux, "The Experience of a
UPS Company in Advanced Battery Monitoring",
MGE UPS Systems, Grenoble, France,'http://wwwmerlingerin.eunet.fr/news/techpap/tp02us.ht'

\bibitem{ZS_jayne}
M.G. Jayne and  C. Morgan, "A New
Mathematical Model of a Lead Acid Battery for
Electric Vehicles", Eighth Int'l Electric Vehicle


\bibitem{ZS_sims}
R.I. Sims, J.C. Carnes, M.A. Dzieciuch and
J.E. Fenton, "Computer Modeling of Automotive
Lead Acid Batteries", Ford Research Laboratories
Technical Report SR-90- 154, Sept., 25, I990

\bibitem{ZS_gig}
R. Giglioli, A. Buonarota, P. Menga, and M. Ceraolo, "Charge and Discharge Fourth
Order Dynamic Model of the Lead Battery", 10th Int'l
Electric Vehicle Symposium, Hong Kong, 1990, p. 1-9.

\bibitem{ZS_rob}
T. Robbins and J. Hawkins "Battery model for
Overcurrent Protection Simulation of DC Distribution Systems", INTELEC 16th
International Telecommunications Energy
Conference, p307--14, 1994.

\bibitem{ZS_zi}
Ziyad M. Salameh, Margaret A. Casacca and William
A. Lynch, "A Mathematical Model for Lead-Acid
Batteries", IEEE Trans. on Energy Conversion. Vol. 7,
No. 1, March 1992

\bibitem{ZS_mar}
Margaret A. Casacca and Ziyad M. Salameh,
"Determination of Lead-Acid Battery Capacity
Via Mathematical Modeling Techniques", IEEE
Trans. on Energy Conversion. Vol. 7, No. 3, Sept.
1992

\bibitem{ZS_chan}
H. L. Chan "A new battery model for use with battery energy storage systems and electric vehicles power systems." Power Engineering Society Winter Meeting, 2000. IEEE. Vol. 1. IEEE, 2000.


\bibitem{ZS_den}
D. W. Dennis, V. S. Battaglia, and A. Belanger, “Electrochemical
modeling of lithium polymer batteries,” J. Power Sources, vol. 110,
no. 2, pp. 310-320, 2002.

\bibitem{ZS_chen}
M. Chen and G. A. Rincon-Mora, “Accurate electrical battery model
capable of predicting runtime and I-V performance,” IEEE Trans.
Energy Conversion, vol. 21m no. 2, pp. 504-511. Jun. 2006.

\bibitem{ZS_kro}
Kroeze, Ryan C., and Philip T. Krein. "Electrical battery model for use in dynamic electric vehicle simulations." Power Electronics Specialists Conference, 2008. PESC 2008. IEEE. IEEE, 2008.

%\bibitem{ZS_bai}
%Baisden, Andrew C., and Ali Emadi. "ADVISOR-based model of a battery and an ultra-capacitor energy source for hybrid electric vehicles." IEEE transactions on vehicular technology 53.1 (2004): 199-205.

\bibitem{ZS_log}
D. L. Logue and P. T. Krein, “Dynamic hybrid electric vehicle
simulation, version 1.0,” University of Illinois, Technical Report
UILUENG-98-0409, December 1998.

\bibitem{ZS_amr}
M. Amrhein and P. T. Krein, “Dynamic simulation for analysis of
hybrid electric vehicle system and subsystem interactions, including
power electronics,” IEEE Trans. Vehicular Tech., vol. 54, no. 3, pp.
825-836, May 2005.

\bibitem{ZS_wip}
K. B. Wipke, M. R. Cuddy, and S. D. Burch, “ADVISOR 2.1: A
user-friendly advanced powertrain simulation using a combined backward/forward
approach,

\bibitem{ZS_adv}
The ADVISOR Code and Manual are Available from the National Renewable
Energy Lab. [Online]. Available: http://www.ctts.nrel.gov/analysis

\bibitem{ZS_col}
G. Cole, “Simple electric vehicle simulation (SIMPLEV) v3.1,” DOE
Idaho National Eng. Lab.

\bibitem{ZS_bun}
D. L. Buntin and J. W. Howze, “A switching logic controller for a
hybrid electric/ICE vehicle,” in Proc. American Control Conf., Seattle,
WA, June 1995, pp. 1169–1175.

\bibitem{ZS_ste}
K. M. Stevens, “A versatile computer model for the design and analysis
of electric and hybrid drive trains,” Master’s thesis, Texas A and M Univ.,
College Station, 1996.

\bibitem{ZS_but}
K. L. Butler, K. M. Stevens, and M. Ehsani, “A versatile computer
simulation tool for design and analysis of electric and hybrid drive
trains,” in 1997 SAE Proc. Electric and Hybrid Vehicle Design Studies,
Detroit, MI, Feb. 1997, pp. 19–25.

%\bibitem{ZS_meh}
%Butler, Karen L., Mehrdad Ehsani, and Preyas Kamath. "A Matlab-based modeling and simulation package for electric and hybrid electric vehicle design." IEEE Transactions on vehicular technology 48.6 (1999): 1770-1778.

\bibitem{ZS_wal}
W. W. Marr and W. J. Walsh, “Life-cycle cost evaluations of electric/
hybrid vehicles,” Energy Conversion Management, vol. 33, no. 9,
pp. 849–853, 1992.

\bibitem{ZS_bum}
J. R. Bumby et al., “Computer modeling of the automotive energy requirements
for internal combustion engine and battery electric-powered
vehicles,” Proc. Inst. Elect. Eng., vol. 132, pt. A, no. 5, pp. 265–279,
1985.

\bibitem{ZS_noo}
R. Noons, J. Swann, and A. Green, “The use of simulation software
to assess advanced powertrains and new technology vehicles,” in Proc.
Electric Vehicle Symp. 15, Brussels, Belgium, Oct. 1998.

\bibitem{ZS_aue}
B. Auert, C. Cheny, B. Raison, and A. Berthon, “Software tool for the
simulation of the electromechanical behavior of a hybrid vehicle,” in
Proc. Electric Vehicle Symp. 15, Brussels, Belgium, Oct. 1998.

\bibitem{ZS_kri}
C. Kricke and S. Hagel, “A hybrid electric vehicle simulation model
for component design and energy management optimization,” in Proc.
FISITA World Automotive Congress, Paris, France, Sept. 1998.

\bibitem{ZS_lu}
Lu, Languang, et al. "A review on the key issues for lithium-ion battery management in electric vehicles." Journal of power sources 226 (2013): 272-288.

\bibitem{ZS_bha}
Bhangu, Bikramjit S., et al. "Nonlinear observers for predicting state-of-charge and state-of-health of lead-acid batteries for hybrid-electric vehicles." IEEE transactions on vehicular technology 54.3 (2005): 783-794.

\bibitem{ZS_pil}
Piller, Sabine, Marion Perrin, and Andreas Jossen. "Methods for state-of-charge determination and their applications." Journal of power sources 96.1 (2001): 113-120.

\bibitem{ZS_cau}
O. Caumont, Ph. Le Moigne, P. Lenain, and C. Rombaut, “An optimized
state of charge algorithm for lead-acid batteries in electric vehicles,” in
Electric Vehicle Symp. , vol. EVS-15, Brussels, Belgium, Sep-Oct. 30–3,
1998. Proceedings on CD-ROM

\bibitem{ZS_cun}
A. B. da Cunha, B. R. de Almeida, and D. C. da SilvaJr., “Remainingcapacity  measurement  and  analysis  of  alkaline  batteries  for  wirelesssensor nodes,”IEEE Transactions on Instrumentation and Measurement,vol. 58, no. 6, pp. 1816–1822, June 2009.


\bibitem{ZS_yu}
Yu, Jianbo. "State-of-health monitoring and prediction of lithium-ion battery using probabilistic indication and state-space model." IEEE Transactions on Instrumentation and Measurement 64.11 (2015): 2937-2949.

\bibitem{ZS_pas}
P. E. Pascoe and A. H. Anbuky, “Standby power system VRLA battery
reserve life estimation scheme,” IEEE Trans. Energy Convers., vol. 20,
no. 4, pp. 887–895, Dec. 2005.

\bibitem{ZS_ver}
M.  Verbrugge  and  E.  Tate,  “Adaptive  state  of  charge  algorithm  for nickel metal hydride batteries including hysteresis phenomena,”Journal of Power Sources, vol. 126, no. 1, pp. 236–249, 2004.

\bibitem{ZS_sin}
 P. Singh, C. Fennie, D. Reisner, and A. Salkind, “A fuzzy logic approach to  state-of-charge  determination  in  high  performance  batteries  with applications  to  electric  vehicles,”  in15th  Electric  Vehicle  Symposium,vol. 15, 1998, pp. 30–35.

\bibitem{ZS_ang}
  C. Chan, E. Lo, and S. Weixiang, “The available capacity computation model  based  on  artificial  neural  network  for  lead–acid  batteries  in electric vehicles,”Journal of Power Sources, vol. 87, no. 1, pp. 201–204,2000.

\bibitem{ZS_11}
 Y.  Lee,  W.  Wang,  and  T.  Kuo,  “Soft  computing  for  battery  state-of-charge (bsoc) estimation in battery string systems,”IEEE Transactionson Industrial Electronics,, vol. 55, no. 1, pp. 229–239, 2008.

\bibitem{ZS_12}
 P.  Pascoe  and  A.  Anbuky,  “Vrla  battery  discharge  reserve  time  estimation,”Power  Electronics,  IEEE  Transactions  on,  vol.  19,  no.  6,  pp.1515–1522, 2004.

\bibitem{ZS_13}
 I. Kim, “Nonlinear state of charge estimator for hybrid electric vehicl ebattery,”IEEE  Transactions  on  Power  Electronics,,  vol.  23,  no.  4,  pp.2027–2034, 2008.

\bibitem{ZS_14}
 Y. He, W. Liu, and B. Koch, “Battery algorithm verification and development using hardware-in-the-loop testing,”Journal of Power Sources,vol. 195, no. 9, pp. 2969–2974, 2010.

\bibitem{ZS_15}
 G.  Plett,  “Extended  kalman  filtering  for  battery  management  systems of lipb-based hev battery packs: Part 1. background,”Journal of power sources, vol. 134, no. 2, pp. 262–276, 2004.

\bibitem{ZS_16}
G. Plett, “Extended  kalman  filtering  for  battery  management  systems  of lipb-based  hev  battery  packs:  Part  2.  modeling  and  identification,”Journal of power sources, vol. 134, no. 2, pp. 262–276, 2004. 

\bibitem{ZS_17}
G. Plett, “Extended  kalman  filtering  for  battery  management  systems  of lipb-based  hev  battery  packs:  Part  3.  state  and  parameter  estimation,”Journal of power sources, vol. 134, no. 2, pp. 277–292, 2004.

\bibitem{ZS_18}
 G. Dong, J. Wei, and Z. Chen, “Kalman filter for onboard state of charge estimation and peak power capability analysis of lithium-ion batteries,”Journal of Power Sources, vol. 328, pp. 615–626, 2016.

\bibitem{ZS_19}
Z.  Chen,  Y.  Fu,  and  C.  Mi,  “State  of  charge  estimation  of  lithium-ionbatteries in electric drive vehicles using extended kalman filtering,”IEEETransactions  on  Vehicular  Technology,,  vol.  62,  no.  3,  pp.  1020–1030, 2013.

\bibitem{ZS_20}
 G.  Dong,  J.  Wei,  C.  Zhang,  and  Z.  Chen,  “Online  state  of  chargeestimation  and  open  circuit  voltage  hysteresis  modeling  of  lifepo  4battery  using  invariant  imbedding  method,”Applied  Energy,  vol.  162,pp. 163–171, 2016.

\bibitem{ZS_21}
  X. Tang, Y. Wang, and Z. Chen, “A method for state-of-charge estimationof lifepo 4 batteries based  on a dual-circuit state observer,”Journal ofPower Sources, vol. 296, pp. 23–29, 2015.

\bibitem{ZS_22}
  Y.  Wang,  C.  Zhang,  and  Z.  Chen,  “A  method  for  state-of-chargeestimation  of  lifepo  4  batteries  at  dynamic  currents  and  temperaturesusing particle filter,”Journal of power sources, vol. 279, pp. 306–311, 2015.

\bibitem{ZS_23}
  X.  Liu,  J.  Wu,  C.  Zhang,  and  Z.  Chen,  “A  method  for  state  ofenergy  estimation  of  lithium-ion  batteries  at  dynamic  currents  andtemperatures,”Journal of Power Sources, vol. 270, pp. 151–157, 2014.

\bibitem{ZS_24}
  L. Zhong, C. Zhang, Y. He, and Z. Chen, “A method for the estimation  of  the  battery  pack  state  of  charge  based  on  in-pack  cells  uniformityanalysis,”Applied Energy, vol. 113, pp. 558–564, 2014.
  
 \bibitem{ZS_shaheer} 
  Muhammad, Shaheer, et al. "A Robust Algorithm for State-of-Charge Estimation with Gain Optimization." IEEE Transactions on Industrial Informatics (2017).
  
  \bibitem{ZS_coi}
  https://www.coilwindingexpo.com/berlin/design-and-test-perfect-the-electric-vehicle-balancing-act
  
  \bibitem{ZS_war}
  R. A. Waraich, M. Galus, C. Dobler, M. Balmer, G. Andersson, and K.
Axhausen, “Plug-in hybrid electric vehicles and smart grid: Investigations
based on a micro-simulation,” Institute for Transport Planning
and Systems, ETH Zurich, Switzerland, Tech. Rep. 10.3929/ethz-a-
005916811, 2009.

\bibitem{ZS_rah}
S. Rahman and Y. Teklu, “Role of the electric vehicle as a distributed
resource,” in Proc. IEEE Power Eng. Soc. Winter Meet., Singapore,
Jan. 23–27, 2000, vol. 1, pp. 528–533.

\bibitem{ZS_gal}
 M. D. Galus and G. Andersson, “Demand management of grid connected
plug-in hybrid electric vehicles (PHEV),” in Proc. IEEE Energy
2030 Conf., Atlanta, GA, Nov. 17–18, 2008, pp. 1–8.

\bibitem{ZS_sun}
Sundstrom, Olle, and Carl Binding. "Flexible charging optimization for electric vehicles considering distribution grid constraints." IEEE Transactions on Smart Grid 3.1 (2012): 26-37.

\bibitem{ZS_sun2}
Sundstrom, Olle, and Carl Binding. "Optimization methods to plan the charging of electric vehicle fleets." Proceedings of the international conference on control, communication and power engineering. 2010.

\bibitem{ZS_salameh}
Salameh, Ziyad M., Margaret A. Casacca, and William A. Lynch. "A mathematical model for lead-acid batteries." IEEE Transactions on Energy Conversion 7.1 (1992): 93-98.

