
\bibitem{JX_1}
Z. Abdin, M. Alim, R. Saidur, M. Islam, W. Rashmi, S. Mekhilef, and A. Wadi. Solar energy harvesting with the application of nanotechnology.  Renewable and sustainable energy reviews, 26:837--852, 2013.

\bibitem{JX_2} S. Agrawal and S. Singh. Multi-port converter for solar powered hybrid vehicle. In Photovoltaic Specialists Conference (PVSC), 2016 IEEE 43rd, pages 3258--3262. IEEE, 2016.

\bibitem{JX_3} N. Amati, A. Canova, F. Cavalli, S. Carabelli, A. Festini, A. Tonoli, and G. Caviasso. Electromagnetic shock absorbersfor automotive suspensions: electromechanical design. In Titolo volume non avvalorato, 2006.

\bibitem{JX_4} S. S. Aqeel and N. E. Elizabeth. An optimized time bounded routing on solar based vehicuar ad-hoc networks using particle swarm optimization. In Intelligent Systems and Control (ISCO), 2016 10th International Conference on, pages 1--5. IEEE, 2016.

\bibitem{JX_5} I. Arsie, M. Cacciato, A. Consoli, G. Petrone, G. Rizzo, M. Sorrentino, and G. Spagnuolo. Hybrid vehicles and solar energy: a possible marriage? ICAT06, Istanbul,
November, 17, 2006.

\bibitem{JX_6} I. Arsie, G. Rizzo, and M. Sorrentino. Optimal design
and dynamic simulation of a hybrid solar vehicle.
Technical report, SAE Technical Paper, 2006.

\bibitem{JX_7} N. Baatar and S. Kim. A thermoelectric generator replacing radiator for internal combustion engine vehicles. TELKOMNIKA (Telecommunication Computing
Electronics and Control), 9(3):523--530, 2013.

\bibitem{JX_8} H. D. Battista, R. J. Mantz, and F. Garelli. Power conditioning for a windchydrogen energy system. Journal
of Power Sources, 155(2):478--486, 2006.

\bibitem{JX_9} D. Bechrakis, E. McKeogh, and P. Gallagher. Simulation and operational assessment for a small autonomous windchydrogen energy system. Energy Conversion and Management, 47(1):46--59, 2006.

\bibitem{JX_10} J. Bourgeois, S. Foell, G. Kortuem, B. A. Price, J. Van Der Linden, E. Y. Elbanhawy, and C. Rimmer. Harvesting green miles from my roof: an investigation into self-sufficient mobility with electric vehicles. In Proceedings of the 2015 ACM International Joint Conference on Pervasive and Ubiquitous Computing, pages
1065--1076. ACM, 2015.

\bibitem{JX_11} J. Cao and A. Emadi. A new battery/ultracapacitor
hybrid energy storage system for electric, hybrid, and plug-in hybrid electric vehicles. IEEE Transactions on Power Electronics, 27(1):122--132, Jan 2012.

\bibitem{JX_12} J. Cao, N. Schofield, and A. Emadi. Battery balancing methods: A comprehensive review. In 2008 IEEE Vehicle Power and Propulsion Conference, pages 1-- 6, Sept 2008.

\bibitem{JX_13} C. Chellaswamy and R. Ramesh. Green energy harvesting: Recharging electric vehicle for pollution free environment. In 2014 International Conference on Smart Structures and Systems (ICSSS), pages 59--66, Oct 2014.

\bibitem{JX_14} Y.-S. Chen, Y.-W. Lin, and C.-H. Wang. A green time-bounded routing protocol in solar-based vehicular networks. In Innovative Mobile and Internet Services in Ubiquitous Computing (IMIS), 2013 Seventh International Conference on, pages 336--341. IEEE, 2013.

\bibitem{JX_15} B. Delfino and F. Fornari. Modeling and control of an integrated fuel cell-wind turbine system. In 2003 IEEE Bologna Power Tech Conference Proceedings,, volume 2, pages 6 pp. Vol.2--, June 2003.

\bibitem{JX_16} J. Dixon. Energy storage for electric vehicles. In 2010 IEEE International Conference on Industrial Technology, pages 20--26, March 2010.

\bibitem{JX_17} E. N. Elizabeth and S. A. Shahul. Velocity optimized solar-time bound traffic free routing (vosttr) ee protocol for vanet’s. International Journal of Computer Science and Information Security, 14(12):438, 2016.

\bibitem{JX_18} N. Espinosa, M. Lazard, L. Aixala, and H. Scherrer. Modeling a thermoelectric generator applied to diesel
automotive heat recovery. Journal of Electronic mate-
rials, 39(9):1446--1455, 2010.

\bibitem{JX_19} Z. Fang, X. Guo, L. Xu, and H. Zhang. Experimental
study of damping and energy regeneration characteristics of a hydraulic electromagnetic shock absorber. Advances in Mechanical Engineering, 5:943528, 2013.

%\bibitem{JX_20} Z. Fang, X. Guo, L. Xu, and H. Zhang. Experimental study of damping and energy regeneration characteristics of a hydraulic electromagnetic shock absorber. Advances in Mechanical Engineering, 5:943528, 2013.

\bibitem{JX_21} V. Fthenakis, J. E. Mason, and K. Zweibel. The technical, geographical, and economic feasibility for solar energy to supply the energy needs of the us. Energy Policy, 37(2):387--399, 2009.

\bibitem{JX_22} A. Gupta, J. A. Jendrzejczyk, T. M. Mulcahy, and J. R. Hull. Design of electromagnetic shock absorbers. International Journal of Mechanics and Materials in Design, 3(3):285--291, Sep 2006.

\bibitem{JX_23} B. L. J. Gysen, T. P. J. van der Sande, J. J. H. Paulides, and E. A. Lomonova. Efficiency of a regenerative direct-drive electromagnetic active suspension. IEEE Transactions on Vehicular Technology, 60(4):1384-- 1393, May 2011.

\bibitem{JX_24} L. Hao and C. Namuduri. Electromechanical regenerative actuator with fault tolerance capability for automotive chassis applications. In 2011 IEEE Energy Conversion Congress and Exposition, pages 1750-- 1757, Sept 2011.

\bibitem{JX_25} R. Hebner, J. Beno, and A. Walls. Flywheel batteries come around again. IEEE Spectrum, 39(4):46--51, Apr 2002.

\bibitem{JX_26} A. Hopkins, N. McNeill, P. Anthony, and P. Mellor. High efficiency bidirectional 5kw dc-dc converter with super-junction mosfets for electric vehicle supercapacitor systems. In Energy Conversion Congress and Exposition (ECCE), 2015 IEEE, pages 4632--4639. IEEE, 2015.

\bibitem{JX_27} C.-T. Hsu, D.-J. Yao, K.-J. Ye, and B. Yu. Renewable energy of waste heat recovery system for automobiles. Journal of Renewable and Sustainable Energy, 2(1):013105, 2010.

\bibitem{JX_28} B. Huang, C.-Y. Hsieh, F. Golnaraghi, and M. Moallem. Development and optimization of an energy-regenerative suspension system under stochastic road excitation. Journal of Sound and Vibration, 357(Supplement C):16--34, 2015.

\bibitem{JX_29} N. Javani, I. Dincer, and G. Naterer. Thermodynamic analysis of waste heat recovery for cooling systems in hybrid and electric vehicles. Energy, 46(1):109--116, 2012.

\bibitem{JX_30} R. P. Kepner. Hydraulic power assist-- a demonstration of hydraulic hybrid vehicle regenerative braking in a road vehicle application. Portugal: Polytechnic Institute of Porto (IPP), page 8, 11 2002.

\bibitem{JX_31} A. Khaligh and Z. Li. Battery, ultracapacitor, fuel cell, and hybrid energy storage systems for electric, hybrid electric, fuel cell, and plug-in hybrid electric vehicles: State of the art. IEEE Transactions on Vehicular
Technology, 59(6):2806--2814, July 2010.

\bibitem{JX_32} M. Khan and M. Iqbal. Dynamic modeling and simulation of a small windcfuel cell hybrid energy system.
Renewable Energy, 30(3):421--439, 2005.

\bibitem{JX_33} P. Khaokhajorn, S. Samipak, S. Nithithanasilp, M. Tanticharoen, and A. Amnuaykanjanasin. Production and secretion of naphthoquinones is mediated by the mfs transporter mfs1 in the entomopathogenic fungus ophiocordyceps sp. bcc1869. World Journal Of Mi-
crobiology Biotechnology, 31(10):1543--1554, 2015. 

\bibitem{JX_34} J. Kim, Y. Wang, M. Pedram, and N. Chang. Fast photovoltaic array reconfiguration for partial solar powered vehicles. In Proceedings of the 2014 international symposium on Low power electronics and design, pages
357--362. ACM, 2014.

\bibitem{JX_35} S. Kim, S. Park, S. Kim, and S.-H. Rhi. A thermo-
electric generator using engine coolant for light-duty internal combustion engine-powered vehicles. Journal of electronic materials, 40(5):812, 2011.

\bibitem{JX_36} S.-K. Kim, B.-C. Won, S.-H. Rhi, S.-H. Kim, J.-H. Yoo, and J.-C. Jang. Thermoelectric power generation system for future hybrid vehicles using hot exhaust gas. Journal of electronic materials, 40(5):778--783, 2011.

\bibitem{JX_37} J. LaGrandeur, D. Crane, S. Hung, B. Mazar, and A. Eder. Automotive waste heat conversion to electric power using skutterudite, tags, pbte and bite. In Thermoelectrics, 2006. ICT’06. 25th International Conference on, pages 343--348. IEEE, 2006.

\bibitem{JX_38} P. Li and L. Zuo. Equivalent circuit modeling of vehicle dynamics with regenerative shock absorbers. International Design Engineering Technical Conferences and Computers and Information in Engineering Conference, 1, 2013.

\bibitem{JX_39} Z. Li, L. Zuo, J. Kuang, and G. Luhrs. Energyharvesting shock absorber with a mechanical motion rectifier. Smart Materials and Structures, 22(2):025008, 2013.

\bibitem{JX_40} Z. Li, L. Zuo, G. Luhrs, L. Lin, and Y. x. Qin. Electro-magnetic energy-harvesting shock absorbers: Design, modeling, and road tests. IEEE Transactions on Vehicular Technology, 62(3):1065--1074, March 2013.

\bibitem{JX_41} F. Liu, S. Duan, F. Liu, B. Liu, and Y. Kang. A variable step size inc mppt method for pv systems. IEEE Transactions on industrial electronics, 55(7):2622-- 2628, 2008.

\bibitem{JX_42} S. Liu, H. Wei, and W. Wang. Investigation on some key issues of regenerative damper with rotary motor for automobile suspension. In Proceedings of 2011 International Conference on Electronic Mechanical Engineering and Information Technology, volume 3, pages 1435--1439, Aug 2011.

\bibitem{JX_43} X. Liu, Y. Deng, Z. Li, and C. Su. Performance analysis of a waste heat recovery thermoelectric generation system for automotive application. Energy Conversion and Management, 90:121--127, 2015.

\bibitem{JX_44} X. Liu, Y. Deng, K. Zhang, M. Xu, Y. Xu, and C. Su. Experiments and simulations on heat exchangers in thermoelectric generator for automotive application.
Applied Thermal Engineering, 71(1):364--370, 2014.

\bibitem{JX_45} S. M. Lukic, J. Cao, R. C. Bansal, F. Rodriguez, and A. Emadi. Energy storage systems for automotive applications. IEEE Transactions on Industrial Elec-
tronics, 55(6):2258--2267, June 2008.

\bibitem{JX_46} M. Lv, N. Guan, Y. Ma, D. Ji, E. Knippel, X. Liu,
and W. Yi. Speed planning for solar-powered electric vehicles. In Proceedings of the Seventh International Conference on Future Energy Systems, page 6. ACM, 2016.

\bibitem{JX_47} X. Ma, Y. Sun, J. Wu, and S. Liu. The research on the algorithm of maximum power point tracking in photo voltaic array of solar car. In Vehicle Power and Propulsion Conference, 2009. VPPC’09. IEEE, pages 1379--1382. IEEE, 2009.

\bibitem{JX_48} D. MacKay. Sustainable energy without the hot air. Design and Technology Education: an International Journal, 16(1), 2011.

\bibitem{JX_49} A. Maravandi and M. Moallem. Regenerative shock
absorber using a two-leg motion conversion mechanism. IEEE/ASME Transactions on Mechatronics, 20(6):2853--2861, Dec 2015.

\bibitem{JX_50} B. Moshfegh. World renewable energy congress sweden; 8-13 may; 2011; linkoping; sweden. In World Renewable Energy Congress Sweden; 8-13 May; 2011; Linkoping; Sweden, volume 10. Linkoping University Electronic Press; Linkopings universitet, 2011.

\bibitem{JX_51} T. Moustafa and W. Moreno. Ram air and wind energy harvesting survey for electrical vehicles and transportation. In SoutheastCon 2017, pages 1--7, March 2017.

\bibitem{JX_51_1} M.~Nasri, I.~B{\"u}rger, S.~Michael, and H.~E. Friedrich.
Waste heat recovery for fuel cell electric vehicle with
  thermochemical energy storage.
In Ecological Vehicles and Renewable Energies (EVER), 2016
  Eleventh International Conference on, pages 1--6. IEEE, 2016.

\bibitem{JX_52} N. K. Navin and R. Sharma. A modified differential evolution approach to phev integrated thermal unit commitment proble. In 2016 IEEE 7th Power India International Conference (PIICON), pages 1--4, Nov 2016.

\bibitem{JX_53} O. Onar, M. Uzunoglu, and M. Alam. Dynamic modeling, design and simulation of a wind/fuel cell/ultracapacitor-based hybrid power generation system. Journal of Power Sources, 161(1):707--722, 2006.

\bibitem{JX_54} B. Orr, A. Akbarzadeh, and P. Lappas. Predicting the performance of a car exhaust heat recovery system that utilises thermoelectric generators and heat pipes. In Solar 2014, pages 645--653. Australian Solar Council, 2014.

\bibitem{JX_55} B. Orr, A. Akbarzadeh, M. Mochizuki, and R. Singh. A review of car waste heat recovery systems utilising thermoelectric generators and heat pipes. Applied Thermal Engineering, 101:490--495, 2016.

\bibitem{JX_56} B. G. Orr and A. Akbarzadeh. Experimental testing of a car exhaust heat recovery system utilising tegs and heat pipes. SAE-A Vehicle Technology Engineer Journal, 2(1), 2016.

\bibitem{JX_57} N. A. Rajan, K. D. Shrikant, B. Dhanalakshmi, and N. Rajasekar. Solar pv array reconfiguration using the concept of standard deviation and genetic algorithm. Energy Procedia, 117:1062--1069, 2017.

\bibitem{JX_58} V. S and F. H. Braking energy regeneration using hydraulic systems. Portugal: Polytechnic Institute of
Porto (IPP), page 8, 2008.

\bibitem{JX_59} R. Sabzehgar, A. Maravandi, and M. Moallem. Energy
regenerative suspension using an algebraic screw linkage mechanism. IEEE/ASME Transactions on Mechatronics, 19(4):1251--1259, Aug 2014.

\bibitem{JX_60} K. M. Saqr, M. K. Mansour, and M. N. Musa. Thermal design of automobile exhaust based thermoelectric generators: Objectives and challenges. International Journal of Automotive Technology, 9(2):155--160, 2008.

\bibitem{JX_61} L. Segel and X. Lu. Vehicular resistance to motion as influenced by road roughness and highway alignment. Australian road research, 12(4):211--222, 1982.

\bibitem{JX_62} T. Senjyu, S. Chakraborty, A. Y. Saber, H. Toyama, A. Yona, and T. Funabashi. Thermal unit commitment strategy with solar and wind energy systems using genetic algorithm operated particle swarm optimization. In 2008 IEEE 2nd International Power and Energy Conference, pages 866--871, Dec 2008.

\bibitem{JX_63} R. Singh, M. K. Gaur, and C. S. Malvi. Study of solar energy operated hybrid mild cars: A review. International Journal of Scientific Engineering and Technology, 1(4):139--148, 2012.

\bibitem{JX_64} J. Storey, P. R. Wilson, and D. Bagnall. The optimized string dynamic photovoltaic array. IEEE Transactions on Power Electronics, 29(4):1768--1776, 2014.

\bibitem{JX_65} S. F. Tie and C. W. Tan. A review of energy sources and energy management system in electric vehicles. Renewable and Sustainable Energy Reviews, 20(Complete):82--102, 2013.

\bibitem{JX_66} P. d. S. Vicente, T. C. Pimenta, and E. R. Ribeiro. Photovoltaic array reconfiguration strategy for maximization of energy production. International Journal of Photoenergy, 2015, 2015.

\bibitem{JX_67} V. V. Vincent and S. Kamalakkannan. Advanced hybrid system for solar car. In Computation of Power, Energy, Information and Communication (ICCPEIC), 2013 International Conference on, pages 21--27. IEEE, 2013.

\bibitem{JX_68} Y. Wang, X. Lin, N. Chang, and M. Pedram. Dynamic reconfiguration of photovoltaic energy harvesting system in hybrid electric vehicles. In Proceedings of the 2012 ACM/IEEE international symposium on Low power electronics and design, pages 109--114. ACM, 2012.

\bibitem{JX_69} X. Xie and Q. Wang. Energy harvesting from a vehicle suspension system. Energy, 86(Supplement C):385--392, 2015.

\bibitem{JX_70} M. K. Yoong, Y. H. Gan, G. D. Gan, C. K. Leong, Z. Y. Phuan, B. K. Cheah, and K. W. Chew. Studies of regenerative braking in electric vehicle. In 2010 IEEE Conference on Sustainable Utilization and Development in Engineering and Technology, pages 40-- 45, Nov 2010.

\bibitem{JX_71} C. Yu and K. Chau. Thermoelectric automotive waste heat energy recovery using maximum power point tracking. Energy Conversion and Management, 50(6):1506--1512, 2009.

\bibitem{JX_72} G. Zhang, J. Cao, and F. Yu. Design of active and energy-regenerative controllers for dc-motor-based suspension. Mechatronics, 22(8):1124--1134, 2012.

\bibitem{JX_73} Y. Zhang, W. Wang, Y. Kobayashi, and K. Shirai. Remaining driving range estimation of electric vehicle. In 2012 IEEE International Electric Vehicle Conference, pages 1--7, March 2012.

\bibitem{JX_74} Y. Zhang, X. Zhang, M. Zhan, K. Guo, F. Zhao, and Z. Liu. Study on a novel hydraulic pumping regenerative suspension for vehicles. Journal of the Franklin Institute, 352(2):485--499, 2015. Special Issue on Control and Estimation of Electrified vehicles.

\bibitem{JX_75} Z. Zhang, X. Zhang, W. Chen, Y. Rasim, W. Salman, H. Pan, Y. Yuan, and C. Wang. A high-efficiency energy regenerative shock absorber using supercapacitors for renewable energy applications in range extended electric vehicle. Applied Energy, 178(Supplement C):177--188, 2016.

\bibitem{JX_76} K. Zorbas, E. Hatzikraniotis, and K. Paraskevopoulos. Power and efficiency calculation in commercial teg and application in wasted heat recovery in automobile. In Proc. of 5th European Conference on Thermoelectrics, volume 8, page 2007, 2007.

\bibitem{JX_77} L. Zuo, B. Scully, J. Shestani, and Y. Zhou. Design and characterization of an electromagnetic energy harvester for vehicle suspensions. Smart Materials and Structures, 19(4):045003, 2010.

\bibitem{JX_78} L. Zuo and P.-S. Zhang. Energy harvesting, ride comfort, and road handling of regenerative vehicle suspensions. volume 135, page 8, February 2013.
