%\documentclass{article}
\documentclass[onecolumn]{IEEEconf}
%\usepackage{a4wide}
\usepackage{enumitem}
\usepackage[usenames, dvipsnames]{color}
\usepackage{graphicx}
\usepackage{subcaption}
\renewcommand{\figurename}{Fig.}
\renewcommand*{\thetable}{\Roman{table}}

\title{Response to the Reviewers' Comments}
\begin{document}

\maketitle

The authors would like to thank the reviewers for the important comments and suggestions on the technical contents and presentation of our manuscript ``Survey of Low-Power Electric Vehicles: A Design Automation Perspective,'' submitted to IEEE Design and Test. This revised manuscript has been greatly improved thanks to the reviewers’ invaluable advices. We have revised the manuscript faithfully following the reviewers’ comments. We include the newly added or significantly modified parts in the revised version of the paper also in Response to Reviewers. We highlight the important technical content changes and set those parts in a red color in the revised paper. Detailed comments and corresponding corrections are listed below:\\

\setlength{\parindent}{0cm}
%%%%%%%%%%%%%%%%%%%%%%%%%%%%
\textbf{Reviewer 1:}
%%%%%%%%%%%%%%%%%%%%%%%%%%%%
\begin{description}
\item [C1: ] Energy-optimization of EVs is a multi-objective optimization problem subject to route profiles, traffic conditions, primary driving purpose and physical constraints. In addition to the layer-wise discussion on optimization problem and the attempted solutions so far, it would be of great help to also incorporate technical comments on a feasibility of cross-layer optimization framework with full awareness of the energy consumption throughout the life cycle of EVs.\\ 
Moreover, it would be interesting to understand if certain optimization trade-off would emerge when it comes to the life-cycle energy management. For instance, energy management conditioned on the cost of battery life/replacement, preferred travel cost (e.g., acceptable time to arrive) and route profiles. In addition to reviewing  the solutions available so far, I think this survey could reach to a much broader audience if it can be positioned to shed some light on the future research directions.
\item [R1: ] There are several research on battery aging and strategies to extend the battery life for EVs.  Please see the following modification of manuscripts.\\

\underline{We added a new paragraph at the end of Section II-B in the manuscript as:}\\
\textcolor{red}{
\textbf{Propulsion Power Optimization minimizing EV life-cycle cost}\\
A cycle life of an EV battery is largely limited by its chemistry, but the charge/discharge behavior also significantly affects the cycle life [79]. From the EV owners' perspective, the total cost of ownership that includes the EV cost (battery chemistry and capacity -- design-time optimization), electricity usage (runtime power management) and battery aging (residual value and/or replacement cost) are equally important. The total cost of ownership optimization for a dedicated user is a very complicated problem and yet is an open problem. 
%The propulsion power optimization minimizing EV life-cycle cost is very important problem for owners since the battery aging varies depending on the speed planning, and directly linked to cost of maintenance increases.
%It is very complex to solve the optimization problem because we should consider the battery aging mechanism and road information as well as EV powertrain at the same time.\\
%
However, there have been several research practices on battery aging and guidelines to extend the EV battery cycle life~[36], [37]. Related work performed experiments to analyze the battery performance (capacity and allowable peak power) by cycle times under various driving conditions (\textit{e.g.}, the temperature, driving profiles, etc.) Such work proposed guidelines under both non-operating condition and operating condition based on the experimental results.}

~\\

\end{description}

\pagebreak


%%%%%%%%%%%%%%%%%%%%%%%%%%%%
\textbf{Reviewer 2:}
%%%%%%%%%%%%%%%%%%%%%%%%%%%%
\begin{description}
\item [C1: ] The paper discusses about the planning issue for runtime power management with a fixed slope. But, with the assumption of a fixed slope, is dynamic programming based planning necessary?
\item [R1: ] This paper discusses the dynamic programming method when the propulsion power optimization with a various road slope. In case of fixed slope, the problem is defined as finding the energy-optimal vehicle speed planning (acceleration, cruising and deceleration) for a given constant road slope.\\
~\\

\item [C2: ] For the driving profile estimation algorithms, is there some related work about how to mitigate the prediction inaccuracy, and what will be the impact of inaccuracy on the overall EV power management?
\item [R2: ] It is impossible to mitigate the inaccuracy of driving profile prediction because we need to know the actual driving profile from the driving monitoring system. 

The impact of inaccuracy is mainly depend on the inaccuracy of the road slope estimation result because the road slope is one of the main factors affecting the EV propulsion power.\\
~\\

\item [C3: ] There are some doubts about the energy harvesting techniques for EVs, especially for PVs and wind energy. For PV, please provide some guidance on the amount of potential energy generation vs. the power consumption of EV. For wind power, will the power be sufficient? Will it add the drag force of the EV? Will there be an overall benefit?
\item [R3: ] The amount of power generated by PVs is maximum 200 W per square meter under 1 kW input solar power. On the other hands, EV power consumption is average 16 kW when Tesla Model S drives at 55 mph.

For the wind power, there is no additional drag force if the wind turbine is installed at the front of a bumper of the vehicle. Therefore, their is a benefit by installation of the wind turbine. Please read a following paper:\\

\textit{S. Mohd, N. Rosly, R. Jamit, S. Shamsudin, A. Abdullah, ``An Evaluation of Drag Coefficient of Wind Turbine System Installed on Moving Car," Applied Mechanics and Materials, October, 2014.}
~\\

\item [C4: ] Finally, for the grid power management on EV charging, there are some more work on power management in the context of smart grid. Please provide more references and brief introduction of the algorithms used.
\item [R4: ] We summarized several research on the V2G implementations and related algorithms. Please see the following modification of the manuscript.\\

\underline{We modify the second paragraph of Section VII-C in the manuscript as:}\\
\textcolor{red}{The V2G implementation involves frequent and intensive charging and discharging processes. To tackle the complex energy exchange between the power grid and the EVs, the unidirectional spinning reserve V2G algorithm was proposed~[243] to adjust the EV charging rate according to a Preference Operating Point (POP), where the minimal preliminary investment and EV batteries degradation can be achieved. Later, bidirectional V2G technologies~[244]--[246] that simultaneously utilize the Grid-to-Vehicle (G2V) and Vehicle-to-Grid (V2G) operations were proposed as they are more flexible in addressing the requirements of both the power grid and the EVs . Meanwhile, various V2G scheduling strategies~[234], [247], [248] targeted at minimizing the power grid load variance were also discussed. In general, one of the most commonly used mitigations to reduce power grid operation loss while accommodating to the large size of the EVs penetration is to shift this extra load to a valley period or to optimize the available power using the coordinated charging schemes~[249]. On the other hand, researchers proposed to utilize the renewable energy as energy sources for EVs. Compared to classical power plants, the renewable energy sources have higher power energy fluctuation and intermittent. However,}~[241] indicates the wind profile in New York matches electric vehicle charging need very well: the electric vehicles could be charged when power supplied by wind power is the greatest and V2G technology could be use to feedback energy to the grid by wind turbines.
Related work~[242] investigated the potential role of electric vehicles in an electricity network with a high contribution from variable generation such as wind power. The simulation models 1000 individual vehicle entities to represent the behavior of larger numbers of vehicles. 
A stochastic trip generation profile is used to generate realistic journey characteristics. 
Finally, experimental results show that the electric vehicles connected to the grid and discharge make up for intermittent of wind generate power, also bring the owners with a certain economic benefits.
\\
\end{description}

\end{document}



